\chapter{Verlinkung von Tests}\label{linkTest}
\begin{figure}[h!]
	\centering
	\includegraphics[width=1\textwidth]{img/Test.jpg}
	\caption{Ansicht Verlinkung Tab in Test}
\end{figure}

~\\In einem erstellten Test erscheint für alle Benutzer mit Schreibrechten der neue Tab \textit{Verlinken}. Die Funkionalität ist identisch mit der ILIAS-Standard Funktionalität \textit{Verlinken}, die aber nicht direkt aus einem Testobjekt verfügbar ist. Es bestehen zwei Auswahlmöglichkeiten einen Test zu verlinken. \textit{In alle Gruppen verlinken} verlinkt das Testobjekt in alle dem Kurs enthaltenen Gruppen. Dabei wird versucht, das verlinkte Objekt in einem Ordner in jeder Gruppe zu speichern, der den selben Namen trägt, wie der dem Testobjekt übergeordnete Ordner. Liegt das  Objekt nicht in einem Ordner, oder enthalten die Gruppen keinen solchen Ordner, wird das Testobjekt direkt in die Gruppen verlinkt. 
\newpage

\section{Verlinkung von Tests}

\begin{figure}[h!]
	\centering
	\includegraphics[width=.7\textwidth]{img/seq_linkGUI.png}
	\caption{Sequenzdiagramm Test verlinken}
\end{figure}
\begin{figure}[h!]
	\centering
	\includegraphics[width=1\textwidth]{img/linkTest.png}
	\caption{Ansicht Link Test}
\end{figure}

~\\Über einen Test gelangen wir über den Tab Verlinkung zu dem dargestellten Bild. 
Mit der Funktion lässt sich dieser Test in die Gruppen des Kurses, in dem die Übung existiert verlinken/verknüpfen. 

\newpage

\subsection*{in alle Gruppen verlinken}
\begin{itemize}
	\item Sollte dieser Radio Button ausgewählt und auf verlinken geklickt werden, so wird diese Übung in alle im Kurs vorhandenen Gruppen verlinkt. 
\end{itemize}

\subsection*{in bestimmte Gruppen verlinken}
\begin{itemize}
	\item Sollte dieser Radio Button ausgewählt werden, erscheint eine Auswahl aller im Kurs vorhandenen Gruppen mit einer Checkbox
	\item Man kann jetzt einzelne Gruppen via der Checkbox markieren
	\item Klickt man auf \textit{Verlinken}, werden diese Gruppen verlinkt
\end{itemize}

~\\(Achtung: Duplikate werden von diesem Tab ausgeschlossen bzw. es wird nicht doppelt verlinkt, sollte der Test schon in diese Gruppe verlinkt worden sein) 
\clearpage
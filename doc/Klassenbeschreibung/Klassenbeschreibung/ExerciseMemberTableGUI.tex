\section{ExerciseMemberTableGUI}

\subsection*{Beschreibung}
Diese Klasse implementiert den Aufbau der Tabelle im Tab \textit{Gruppenfilter}. 
Sie erbt von der ILIAS-Core Klasse ilTable2GUI.

\subsection*{Klassenvariablen}
\subparagraph{exc}
protected - type: ilExercise - Die darzustellende Exercise
\subparagraph{ass}
protected - type: ilAssignment - Das darzustellende Assignment
\subparagraph{exc\_id}
protected - type: integer - ID der Exercise
\subparagraph{ass\_id}
protected - type: integer - ID des Assignment
\subparagraph{sent\_col}
protected - type: boolean - Ob eine Mail versendet wurde
\subparagraph{selected}
protected - type: Array - Selektierte Reihen
\subparagraph{teams}
protected - type: Array
\subparagraph{group}
protected - type: integer - Gruppen ID der darzustellenden Gruppe


\subsection*{Funktions-Liste}

\subparagraph{\nameref{__constructTable}}
\subparagraph{\nameref{getSelectableColumns}}
\subparagraph{\nameref{fillRow}}
\subparagraph{\nameref{isGroupMember}}

\subsection*{Funktionen}

\subsubsection*{\textit{\_\_construct}}\label{__constructTable}


Konstruktor der Tabelle für \textit{Gruppenfilter}
Erstellt eine gefilterte Tabelle
\subparagraph{Beschreibung}
\begin{itemize}
	\item[]  \noindent\fbox{\_\_construct(\$a\_parent\_obj, \$a\_parent\_cmd, \$a\_exc, \$a\_ass,\$group\_id)}
	\item[] Erstellt die gefilterte Tabelle nach den übergebenen Parametern
\end{itemize}
\subparagraph{Parameter-Liste}
\begin{itemize}
	\item[] \textbf{a\_parent\_obj} - type: object - Objekt in dem die Tabelle erzeugt wird
	\item[] \textbf{a\_parent\_cmd} - type: String - Letztes Command der Klasse die den Konstruktor aufruft
	\item[]\textbf{a\_exc} - type: ilExercise - Exercise die in der Tabelle dargestellt werden soll
	\item[]\textbf{a\_ass} - type: ilAssignment - Assignment nach dem gefiltert wird
	\item[]\textbf{group\_id} - type: integer - Gruppen-ID nach der gefiltert wird
\end{itemize}
\subparagraph{Rückgabewerte}
\begin{itemize}
	\item[] Gibt keine Werte zurück
\end{itemize}



\subsubsection*{\textit{getSelectableColumns}}\label{getSelectableColumns}


Gibt ein Array zurück, welches bestimmt welche Spalten in der Tabellenansicht aktiviert bzw. deaktiviert werden können.
\subparagraph{Beschreibung}
\begin{itemize}
	\item[]  \noindent\fbox{getSelectableColumns()}
	\item[] Gibt ein Array der selektierbaren Spalten zurück
\end{itemize}
\subparagraph{Parameter-Liste}
\subparagraph{Rückgabewerte}
\begin{itemize}
	\item[] Gibt ein mehrdimensionales Array zurück.
\end{itemize}

\subsubsection*{\textit{fillRow}}\label{fillRow}
Füllt eine Reihe mit den Daten des übergebenen Parameter. Wird von der Methode \nameref{__constructTable} aufgerufen.
\subparagraph{Beschreibung}
\begin{itemize}
	\item[]  \noindent\fbox{fillRow(\$member)}
	\item[] Füllt eine Reihe der Tabelle
\end{itemize}
\subparagraph{Parameter-Liste}
\begin{itemize}
	\item[] \textbf{member} - type: Array - Ein Array mit den Feldern usr\_id, team\_id, team, und name
\end{itemize}
\subparagraph{Rückgabewerte}
\begin{itemize}
	\item[] Gibt keine Werte zurück
\end{itemize}

\subsubsection*{\textit{isGroupMember}}\label{isGroupMember}
\subparagraph{Beschreibung}
\begin{itemize}
	\item[]  \noindent\fbox{isGroupMember(\$member,\$group\_id)}
	\item[] Überprüft auf Gruppenmitgliedschaft 
\end{itemize}
\subparagraph{Parameter-Liste}
\begin{itemize}
	\item[] \textbf{member} - type: Array - Ein Array mit minimum dem Feld usr\_id
	\item[] \textbf{group\_id} - type: integer - Gruppen-ID die überprüft wird
\end{itemize}
\subparagraph{Rückgabewerte}
\begin{itemize}
	\item[] boolean - \textbf{TRUE} falls die User-ID in der Gruppe mit der Gruppen-ID vorhanden ist, \textbf{FALSE} sonst.
\end{itemize}

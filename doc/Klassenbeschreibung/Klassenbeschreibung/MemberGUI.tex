\section{ilACOMemberGUI}

\subsection*{Beschreibung}
Diese Klasse implementiert die Funktion \textit{Mitglieder verschieben} und beinhaltet die Möglichkeit, einzelne Mitglieder einer Gruppe in einem Kurs in eine andere Gruppe zu verschieben. 
Hierzu wird der zu verschiebende Benutzer über ein AutoComplete-Feld ausgewählt, anschließend werden die Gruppen angezeigt, von denen er verschoben werden kann, sowie die Gruppen, in die er verschoben werden kann.

\subsection*{Klassenvariablen}
\subparagraph{ctrl}
protected - type: ilCtrl - Steuerung Zugriffsrechte
\subparagraph{tpl}
protected - type: ilTemplate - Template für Darstellung und Formatierung
\subparagraph{pl}
protected - type: ilACOPlugin - Instanz des Plugins
\subparagraph{tabs}
protected - type: ilTabsGUI - Verwaltung von Tabs
\subparagraph{ilLocator}
protected - type: ilLocatorGUI - Darstellung in der Tree Hierarchie
\subparagraph{lng}
protected - type: ilLanguage - Einbindung der Language-File von ILIAS
\subparagraph{tree}
protected - type: ilTree - Verwaltung der Tree Hierarchie
\subparagraph{course\_id}
protected - type: Integer - Kurs ID für Verwendung in der Klasse
\subparagraph{userLogin}
protected - type: String - Benutzername für Verwendung in der Klasse
\subparagraph{groupTitle}
protected - type: String - Gruppentitel für Gruppe zum Entfernen
\subparagraph{destinationTitle}
protected - type: String - Gruppentitel für Gruppe zum Einfügen
\subparagraph{group\_title}
protected - type: String - Gruppentitel für Gruppe zum Entfernen
\subparagraph{destination\_title}
protected - type: String - Gruppentitel für Gruppe zum Einfügen


\subsection*{Funktions-Liste}
\paragraph{\nameref{constructMGUI}}
\paragraph{\nameref{setTitleAndIconMGUI}}
\paragraph{\nameref{prepareOutputMGUI}}
\paragraph{\nameref{executeCommandMGUI}}
\paragraph{\nameref{viewMGUI}}
\paragraph{\nameref{initFormMGUI}}
\paragraph{\nameref{selectMemberMGUI}}
\paragraph{\nameref{doUserAutoCompleteMGUI}}
\paragraph{\nameref{moveMemberMGUI}}
\paragraph{\nameref{getRoleIDMGUI}}
\paragraph{\nameref{manipulateDBMGUI}}
\paragraph{\nameref{getGroupIdByTitleMGUI}}
\paragraph{\nameref{getMemberIdByLoginMGUI}}
\paragraph{\nameref{checkIfUserExistsInGroupMGUI}}
\paragraph{\nameref{checkIfUserNotExistsInGroupMGUI}}
\paragraph{\nameref{checkIfGroupExistsMGUI}}
\paragraph{\nameref{getGroupsMGUI}}
\paragraph{\nameref{getGroupsWhereMemberMGUI}}
\paragraph{\nameref{getGroupsWhereNotMemberMGUI}}
\paragraph{\nameref{checkAccessMGUI}}

\subsection*{Funktionen}

\subsubsection*{\textit{\_\_construct}}\label{constructMGUI}
\subparagraph{Beschreibung}
\begin{itemize}
	\item[] \noindent\fbox{\_\_construct()} 
	\item[] Konstruktion der Grundinstanz
\end{itemize}

\subsubsection*{\textit{prepareOutput}}\label{prepareOutputMGUI}
\subparagraph{Beschreibung}
\begin{itemize}
	\item[] \noindent\fbox{prepareOutput()}
	\item[] Grundlegende Eigenschaften werden festgelegt: Tabs, Position in der Tree Hierarchie, BackTarget, Titel und Icon, sowie das verwendete Template
\end{itemize}

\subsubsection*{\textit{setTitleAndIcon}}\label{setTitleAndIconMGUI}
\subparagraph{Beschreibung}
\begin{itemize}
	\item[] \noindent\fbox{setTitleAndIcon()} 
	\item[] Titel und Symbol des Objekt werden festgelegt
\end{itemize}

\subsubsection*{\textit{executeCommand}}\label{executeCommandMGUI}
\subparagraph{Beschreibung}
\begin{itemize}
	\item[] \noindent\fbox{executeCommand()} 
	\item[] Regelung der Umsetzung von Befehlen mittels CommandButtons
\end{itemize}

\subsubsection*{\textit{view}}\label{viewMGUI}
\subparagraph{Beschreibung}
\begin{itemize}
	\item[] \noindent\fbox{view()} 
	\item[] Ausgabe der initialen Ansicht als HTML-Dokument
\end{itemize}

\subsubsection*{\textit{initForm}}\label{initFormMGUI}
\subparagraph{Beschreibung}
\begin{itemize}
	\item[] \noindent\fbox{initForm()} 
	\item[] Formular wird generiert und mit initialen Werten gefüllt
\end{itemize}
\subparagraph{Rückgabewerte}
\begin{itemize}
\item[] \textbf{form} - Formular mit Standardwerten, type: ilPropertyFormGUI
\end{itemize}

\subsubsection*{\textit{selectMember}}\label{selectMemberMGUI}
\subparagraph{Beschreibung}
\begin{itemize}
	\item[] \noindent\fbox{selectMember()} 
	\item[] Formularansicht mit Name des Benutzers, Gruppen, in denen er Mitglied bw. kein Mitglied ist. Ruft nach Klicken auf den CommandButon die Funktion \nameref{moveMemberMGUI} auf
\end{itemize}
\subparagraph{Rückgabewerte}
\begin{itemize}
	\item[] \textbf{form} - Formular gemäß dem angegebenen Benutzer, type: ilPropertyFormGUI
\end{itemize}

\subsubsection*{\textit{doUserAutoComplete}}\label{doUserAutoCompleteMGUI}
\subparagraph{Beschreibung}
\begin{itemize}
	\item[] \noindent\fbox{doUserAutoComplete()} 
	\item[] Werden in einem Eingabefeld mind. 3 Zeichen eingegeben, versucht die Funktion Vorschläge zur Auswahl von Benutzern zu machen
\end{itemize}

\subsubsection*{\textit{moveMember}}\label{moveMemberMGUI}
\subparagraph{Beschreibung}
\begin{itemize}
	\item[] \noindent\fbox{moveMember()} 
	\item[] Werte aus dem Formular aus \nameref{selectMemberMGUI} werden ausgewertet und an \nameref{manipulateDBMGUI} übergeben. Anschließend wieder zurück zur initialen Formularansicht.
\end{itemize}

\subsubsection*{\textit{getRoleID}}\label{getRoleIDMGUI}
\subparagraph{Beschreibung}
\begin{itemize}
	\item[] \noindent\fbox{getRoleID(\$description)} 
	\item[] Gibt die ObjektID zurück, die dem Eintrag in der Datenbank entspricht, in dem die Rolle als Gruppenmitglied geregelt ist. 
\end{itemize}
\subparagraph{Parameter-Liste}
\begin{itemize}
	\item[] \textbf{description} - Datenbankeintrag für Regelung Gruppenmitgliedschaft, type: String
\end{itemize}
\subparagraph{Rückgabewerte}
\begin{itemize}
	\item[] \textbf{role\_id} - ObjektID des entsprechenden Datenbankeintrags, type: Integer
\end{itemize}

\subsubsection*{\textit{manipulateDB}}\label{manipulateDBMGUI}
\subparagraph{Beschreibung}
\begin{itemize}
	\item[] \noindent\fbox{manipulateDB(\$member\_id,\$role\_id\_source,\$destination\_id,\$role\_id\_dest,\$source\_id)} 
	\item[] Schreibt die Werte aus dem Formular in die Datenbank und verschiebt somit den Benutzer aus einer Gruppe in eine andere und passt die Zugriffsrechte des Benutzers entsprechend an.
\end{itemize}
\subparagraph{Parameter-Liste}
\begin{itemize}
	\item[] \textbf{member\_id} - UserID des Benutzers, type: Integer
	\item[] \textbf{role\_id\_source} - ID alt zur Festlegung der Rolle des Benutzers, type: Integer
	\item[] \textbf{destination\_id} - ID der neuen Gruppe des Benutzers, type: Integer
	\item[] \textbf{role\_id\_dest} - ID neu zur Festlegung der Rolle des Benutzers, type: Integer
	\item[] \textbf{source\_id} - ID der bisherigen Gruppe des Benutzers, type: Integer
\end{itemize}

\subsubsection*{\textit{getGroupIdByTitle}}\label{getGroupIdByTitleMGUI}
\subparagraph{Beschreibung}
\begin{itemize}
	\item[] \noindent\fbox{getGroupIdByTitle(\$group\_title,\$course\_id)} 
	\item[] Gibt bei Eingabe des Gruppentitels die dazugehörige ObjektID zurück.
\end{itemize}
\subparagraph{Parameter-Liste}
\begin{itemize}
	\item[] \textbf{group\_title} - Titel der Gruppe, type: String
	\item[] \textbf{course\_id} - ReferenzID des übergeordneten Kurses, type: Integer
\end{itemize}
\subparagraph{Rückgabewerte}
\begin{itemize}
	\item[] \textbf{group\_id} - ObjektID der Gruppe, type:
\end{itemize}

\subsubsection*{\textit{getMemberIdByLogin}}\label{getMemberIdByLoginMGUI}
\subparagraph{Beschreibung}
\begin{itemize}
	\item[] \noindent\fbox{getMemberIdByLogin(\$member\_login)} 
	\item[] Gibt bei Eingabe des Benutzernamens die dazugehörige UserID zurück.	
\end{itemize}
\subparagraph{Parameter-Liste}
\begin{itemize}
	\item[] \textbf{member\_login} - Benutzername, type: String
\end{itemize}
\subparagraph{Rückgabewerte}
\begin{itemize}
	\item[] \textbf{member\_id[0]["usr\_id"]} - UserID, type: Integer
\end{itemize}

\subsubsection*{\textit{checkIfUserExistsInGroup}}\label{checkIfUserExistsInGroupMGUI}
\subparagraph{Beschreibung}
\begin{itemize}
	\item[] \noindent\fbox{checkIfUserExistsInGroup(\$member\_id,\$group\_id)} 
	\item[] Überprüft, ob der Benutzer in einer Gruppe bereits vorhanden ist oder nicht und gibt einen entsprechenden Boolean-Wert zurück.
\end{itemize}
\subparagraph{Parameter-Liste}
\begin{itemize}
	\item[] \textbf{member\_id} - UserID des Benutzers, type: 
	\item[] \textbf{group\_id} - ObjektID der Gruppe, type: 
\end{itemize}
\subparagraph{Rückgabewerte}
\begin{itemize}
	\item[] \textbf{(true,false)} - Benutzer bereits Mitglied? (ja, nein), type: Boolean
\end{itemize}

\subsubsection*{\textit{checkIfUserNotExistsInGroup}}\label{checkIfUserNotExistsInGroupMGUI}
\subparagraph{Beschreibung}
\begin{itemize}
	\item[] \noindent\fbox{checkIfUserNotExistsInGroup(\$member\_id,\$group\_id)} 
	\item[] Überprüft, ob der Benutzer in einer Gruppe bereits vorhanden ist oder nicht und gibt einen entsprechenden Boolean-Wert zurück.
\end{itemize}
\subparagraph{Parameter-Liste}
\begin{itemize}
	\item[] \textbf{member\_id} - UserID des Benutzers, type: Integer
	\item[] \textbf{group\_id} - ObjektID der Gruppe, type: Integer
\end{itemize}
\subparagraph{Rückgabewerte}
\begin{itemize}
	\item[] \textbf{(true,false)} - Benutzer kein Mitglied? (ja, nein), type:
\end{itemize}

\subsubsection*{\textit{checkIfGroupExists}}\label{checkIfGroupExistsMGUI}
\subparagraph{Beschreibung}
\begin{itemize}
	\item[] \noindent\fbox{checkIfGroupExists(\$group\_id)} 
	\item[] Überprüft, ob die angegebene Gruppe tatsächlich existiert.
\end{itemize}
\subparagraph{Parameter-Liste}
\begin{itemize}
	\item[] \textbf{\$group\_id} - ObjektID der Gruppe, type: Integer
\end{itemize}
\subparagraph{Rückgabewerte}
\begin{itemize}
	\item[] \textbf{(true,false)} - Gruppe vorhanden? (ja/nein), type: Boolean
\end{itemize}

\subsubsection*{\textit{getGroups}}\label{getGroupsMGUI}
\subparagraph{Beschreibung}
\begin{itemize}
	\item[] \noindent\fbox{getGroups()} 
	\item[] Funktion gibt die Titel aller Gruppen zurück, die in einem Kurs existent sind. 
\end{itemize}
\subparagraph{Rückgabewerte}
\begin{itemize}
	\item[] \textbf{output} - Titel der Gruppen, type: Array
\end{itemize}

\subsubsection*{\textit{getGroupsWhereMember}}\label{getGroupsWhereMemberMGUI}
\subparagraph{Beschreibung}
\begin{itemize}
	\item[] \noindent\fbox{getGroupsWhereMember(\$usr\_id)} 
	\item[] Gibt die Titel der Gruppen zurück, in denen ein bestimmter Benutzer ein Mitglied ist.
\end{itemize}
\subparagraph{Parameter-Liste}
\begin{itemize}
	\item[] \textbf{usr\_id} - UserID des Benutzers, type: Integer
\end{itemize}
\subparagraph{Rückgabewerte}
\begin{itemize}
	\item[] \textbf{output} - Titel der Gruppen, type: Array
\end{itemize}

\subsubsection*{\textit{getGroupsWhereNotMember}}\label{getGroupsWhereNotMemberMGUI}
\subparagraph{Beschreibung}
\begin{itemize}
	\item[] \noindent\fbox{getGroupsWhereNotMember(\$usr\_id)} 
	\item[] Gibt die Titel der Gruppen zurück, in denen ein bestimmter Benutzer kein Mitglied ist.
\end{itemize}
\subparagraph{Parameter-Liste}
\begin{itemize}
	\item[] \textbf{usr\_id} - UserID des Benutzers, type: Integer
\end{itemize}
\subparagraph{Rückgabewerte}
\begin{itemize}
	\item[] \textbf{output} - Titel der Gruppen, type: Array
\end{itemize}

\subsubsection*{\textit{checkAccess}}\label{checkAccessMGUI}
\subparagraph{Beschreibung}
\begin{itemize}
	\item[] \noindent\fbox{checkAccess()} 
	\item[] Überprüft, ob man Lese- oder Schreibrechte auf das aktuelle Objekt hat.
\end{itemize}
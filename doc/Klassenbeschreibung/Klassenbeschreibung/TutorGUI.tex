\section{ilACOTutorGUI}

\subsection*{Beschreibung}
blabla

\subsection*{Klassenvariablen}
\subparagraph{VIEW\_ASSIGNMENT}
protected - type: Integer - Konstante
\subparagraph{VIEW\_PARTICIPANT}
protected - type: Integer - Konstante
\subparagraph{VIEW\_GRADES}
protected - type: Integer - Konstante
\subparagraph{exercise}
protected - type: ilObjExercise - Exercise Objekt
\subparagraph{assignment}
protected - type: ilExAssignment - Ausgewähltes Assignment
\subparagraph{ctrl}
protected - type: ilCtrl - Steuerung Zugriffsrechte
\subparagraph{tpl}
protected - type: ilTemplate - Template für Darstellung und Formatierung
\subparagraph{pl}
protected - type: ilACOPlugin - Instanz des Plugins
\subparagraph{tabs}
protected - type: ilTabsGUI - Verwaltung von Tabs
\subparagraph{ilLocator}
protected - type: ilLocatorGUI - Darstellung in der Tree Hierarchie
\subparagraph{lng}
protected - type: ilLanguage - Einbindung der Language-File von ILIAS
\subparagraph{tree}
protected - type: ilTree - Verwaltung der Tree Hierarchie
\subparagraph{group}
protected - type:  - Gruppen-ID der selektierten Gruppe



\subsection*{Funktions-Liste}
\paragraph{\nameref{constructTGUI}}
\paragraph{\nameref{prepareOutputTGUI}}
\paragraph{\nameref{setTitleAndIconTGUI}}
\paragraph{\nameref{executeCommandTGUI}}
\paragraph{\nameref{getViewBackTGUI}}
\paragraph{\nameref{initSubmissionTGUI}}
\paragraph{\nameref{viewTGUI}}
\paragraph{\nameref{getAssignmentTGUI}}
\paragraph{\nameref{isCourseTGUI}}
\paragraph{\nameref{getParentIdsTGUI}}
\paragraph{\nameref{getGroupsTGUI}}
\paragraph{\nameref{membersObjectTGUI}}
\paragraph{\nameref{downloadAllObjectTGUI}}
\paragraph{\nameref{isGroupMemberTGUI}}
\paragraph{\nameref{saveCommentsObjectTGUI}}
\paragraph{\nameref{saveCommentForLearnersObjectTGUI}}
\paragraph{\nameref{createTeamsObjectTGUI}}
\paragraph{\nameref{dissolveTeamsObjectTGUI}}
\paragraph{\nameref{getMultiActionUserIdsTGUI}}
\paragraph{\nameref{redirectFeedbackMailObjectTGUI}}
\paragraph{\nameref{sendMembersObjectTGUI}}
\paragraph{\nameref{saveStatusAllObjectTGUI}}
\paragraph{\nameref{saveStatusTGUI}}
\paragraph{\nameref{selectAssignmentObjectTGUI}}
\paragraph{\nameref{checkAccessTGUI}}


\subsection*{Funktionen}

\subsubsection*{\textit{\_\_construct}}\label{constructTGUI}
\subparagraph{Beschreibung}
\begin{itemize}
	\item[] \noindent\fbox{\_\_construct()}
	\item[] Konstruktion der Grundinstanz
\end{itemize}

\subsubsection*{\textit{prepareOutput}}\label{prepareOutputTGUI}
\subparagraph{Beschreibung}
\begin{itemize}
	\item[] \noindent\fbox{prepareOutput()}
	\item[] Grundlegende Eigenschaften werden festgelegt: Tabs, Position in der Tree Hierarchie, BackTarget, Titel und Icon, sowie das verwendete Template
\end{itemize}

\subsubsection*{\textit{setTitleAndIcon}}\label{setTitleAndIconTGUI}
\subparagraph{Beschreibung}
\begin{itemize}
	\item[] \noindent\fbox{setTitleAndIcon()}
	\item[] Titel und Symbol des Objekt werden festgelegt
\end{itemize}

\subsubsection*{\textit{executeCommand}}\label{executeCommandTGUI}
\subparagraph{Beschreibung}
\begin{itemize}
	\item[] \noindent\fbox{executeCommand()}
	\item[] Regelung der Umsetzung von Befehlen mittels CommandButtons
\end{itemize}

\subsubsection*{\textit{getViewBack}}\label{getViewBackTGUI}
\subparagraph{Beschreibung}
\begin{itemize}
	\item[] \noindent\fbox{getViewBack()}
	\item[] Gibt das letztes Command für eine Ansicht zurück
\end{itemize}
\subparagraph{Rückgabewerte}
\begin{itemize}
	\item[] \textbf{back\_cmd} - Command , type: String
\end{itemize}

\subsubsection*{\textit{initSubmission}}\label{initSubmissionTGUI}
\subparagraph{Beschreibung}
\begin{itemize}
	\item[] \noindent\fbox{initSubmission()}
	\item[] Erstellt eine leere neue Abgabe für einen Benutzer
\end{itemize}
\subparagraph{Rückgabewerte}
\begin{itemize}
	\item[] \textbf{new ilExSubmission} - Neue Abgabe für den User , type: ilExSubmission
\end{itemize}

\subsubsection*{\textit{view}}\label{viewTGUI}
\subparagraph{Beschreibung}
\begin{itemize}
	\item[] \noindent\fbox{view()}
	\item[] Ausgabe der initialen Ansicht als HTML-Dokument
\end{itemize}

\subsubsection*{\textit{getAssignment}}\label{getAssignmentTGUI}
\subparagraph{Beschreibung}
\begin{itemize}
	\item[] \noindent\fbox{getAssignment(\$ass\_id)}
	\item[] Gibt das Assignment mit der übergebenen ID der aktuellen Exercise zurück
\end{itemize}
\subparagraph{Parameter-Liste}
\begin{itemize}
	\item[] \textbf{as\_id} - Assignment ID , type: Integer
\end{itemize}
\subparagraph{Rückgabewerte}
\begin{itemize}
	\item[] \textbf{as} - Assignment mit übergebener ID, type: ilExAssignment
\end{itemize}

\subsubsection*{\textit{isCourse}}\label{isCourseTGUI}
\subparagraph{Beschreibung}
\begin{itemize}
	\item[] \noindent\fbox{isCourse(\$ref\_id)}
	\item[] Überprüft ob das Object mit der ID \$ref\_id vom Typ Kurs ist
\end{itemize}
\subparagraph{Parameter-Liste}
\begin{itemize}
	\item[] \textbf{ref\_id} - ID des zu überprüfenden Objekts, type: Integer
\end{itemize}
\subparagraph{Rückgabewerte}
\begin{itemize}
	\item[] \textbf{TRUE}, falls das Objekt ein Kurs ist, \textbf{FALSE} sonst, type: Boolean
\end{itemize}

\subsubsection*{\textit{getParentIds}}\label{getParentIdsTGUI}
\subparagraph{Beschreibung}
\begin{itemize}
	\item[] \noindent\fbox{getParentIds(\$id)}
	\item[]  Gibt alle IDs der Objekte zurück in denen das Objekt mit der ID \$id liegt.
\end{itemize}
\subparagraph{Parameter-Liste}
\begin{itemize}
	\item[] \textbf{id} - ID des zu überprüfenden Objekts , type: Integer
\end{itemize}
\subparagraph{Rückgabewerte}
\begin{itemize}
	\item[] \textbf{ids} - IDs aller Übergeordneter Objekte , type: Array
\end{itemize}

\subsubsection*{\textit{getGroups}}\label{getGroupsTGUI}
\subparagraph{Beschreibung}
\begin{itemize}
	\item[] \noindent\fbox{getGroups()}
	\item[] Gibt alle Gruppen des Kurses zurück
\end{itemize}
\subparagraph{Rückgabewerte}
\begin{itemize}
	\item[] \textbf{output} - Alle Gruppen des Kurses als mehrdimensionales Array , type: Array
\end{itemize}


\subsubsection*{\textit{membersObject}}\label{membersObjectTGUI}
\subparagraph{Beschreibung}
\begin{itemize}
	\item[] \noindent\fbox{membersObject()}
	\item[] Erstellt die Standardansicht mit Toolbar und Buttons, ruft die Tabelle auf und übergibt die nötigen Parameter
\end{itemize}


\subsubsection*{\textit{downloadAllObject}}\label{downloadAllObjectTGUI}
\subparagraph{Beschreibung}
\begin{itemize}
	\item[] \noindent\fbox{downloadAllObject()}
	\item[] Lädt alle Abgegebenen Dateien des Aktuellen Gruppenfilter herunter
\end{itemize}

\subsubsection*{\textit{isGroupMember}}\label{isGroupMemberTGUI}
\subparagraph{Beschreibung}
\begin{itemize}
	\item[] \noindent\fbox{isGroupMember(\$member,\$group\_id)}
	\item[]  Überprüft ob ein Benutzer Mitglied in einer Gruppe ist
\end{itemize}
\subparagraph{Parameter-Liste}
\begin{itemize}
	\item[] \textbf{member} - Benutzer ID des zu überprüfenden Benutzers , type: Integer 
	\item[] \textbf{group\_id} - Gruppen ID der zu überprüfenden Gruppe , type: Integer
\end{itemize}
\subparagraph{Rückgabewerte}
\begin{itemize}
	\item[] \textbf{TRUE}, falls der Benutzer Mitglied in der Gruppe ist, \textbf{FALSE} sonst - , type: Boolean
\end{itemize}

\subsubsection*{\textit{saveCommentsObject}}\label{saveCommentsObjectTGUI}
\subparagraph{Beschreibung}
\begin{itemize}
	\item[] \noindent\fbox{saveCommentsObject()}
	\item[] Speichert die eingegebene Notiz für Tutoren
\end{itemize}


\subsubsection*{\textit{saveCommentForLearnersObject}}\label{saveCommentForLearnersObjectTGUI}
\subparagraph{Beschreibung}
\begin{itemize}
	\item[] \noindent\fbox{saveCommentForLearnersObject()}
	\item[] Speichert den eingegebenen Kommentar für den Benutzer
\end{itemize}

\subsubsection*{\textit{createTeamsObject}}\label{createTeamsObjectTGUI}
\subparagraph{Beschreibung}
\begin{itemize}
	\item[] \noindent\fbox{createTeamsObject()}
	\item[] Erstellt ein Team aus allen selektierten Benutzern
\end{itemize}

\subsubsection*{\textit{dissolveTeamsObject}}\label{dissolveTeamsObjectTGUI}
\subparagraph{Beschreibung}
\begin{itemize}
	\item[] \noindent\fbox{dissolveTeamsObject()}
	\item[] löst alle ausgewählten Teams auf
\end{itemize}

\subsubsection*{\textit{getMultiActionUserIds}}\label{getMultiActionUserIdsTGUI}
\subparagraph{Beschreibung}
\begin{itemize}
	\item[] \noindent\fbox{getMultiActionUserIds(\$a\_keep\_teams)}
	\item[] Gibt die Benutzer-IDs aller selektierten Benutzer zurück
\end{itemize}
\subparagraph{Parameter-Liste}
\begin{itemize}
	\item[] \textbf{\$a\_keep\_teams} - , type: Boolean
\end{itemize}
\subparagraph{Rückgabewerte}
\begin{itemize}
	\item[] \textbf{members} - Array mit Benutzer IDs , type: Array
\end{itemize}

\subsubsection*{\textit{redirectFeedbackMailObject}}\label{redirectFeedbackMailObjectTGUI}
\subparagraph{Beschreibung}
\begin{itemize}
	\item[] \noindent\fbox{redirectFeedbackMailObject()}
	\item[]  Setzt den Feedback Status des Benutzers und leitet zum Email Fenster weiter
\end{itemize}

\subsubsection*{\textit{sendMembersObject}}\label{sendMembersObjectTGUI}
\subparagraph{Beschreibung}
\begin{itemize}
	\item[] \noindent\fbox{sendMembersObject()}
	\item[]  Sendet das Assignment per Mail an alle ausgewählten Benutzer
\end{itemize}

\subsubsection*{\textit{saveStatusAllObject}}\label{saveStatusAllObjectTGUI}
\subparagraph{Beschreibung}
\begin{itemize}
	\item[] \noindent\fbox{saveStatusAllObject()}
	\item[]  Ließt alle Daten aus der Tabelle aus
\end{itemize}

\subsubsection*{\textit{saveStatus}}\label{saveStatusTGUI}
\subparagraph{Beschreibung}
\begin{itemize}
	\item[] \noindent\fbox{saveStatus(array \$a\_data,\$user\_ids)}
	\item[]  Speichert alle veränderten Daten aus der Tabelle in die Datenbank. Berechnet die Gesamtpunkte eines Arbeitsblatts neu
\end{itemize}
\subparagraph{Parameter-Liste}
\begin{itemize}
	\item[] \textbf{a\_data} - Daten die gespeichert werden , type: Array
	\item[] \textbf{user\_id} - Benutzer IDs aller veränderten Benutzer , type: Array
\end{itemize}

\subsubsection*{\textit{selectAssignmentObject}}\label{selectAssignmentObjectTGUI}
\subparagraph{Beschreibung}
\begin{itemize}
	\item[] \noindent\fbox{selectAssignmentObject()}
	\item[]  Generiert die Tabellenansicht für die ausgewählte Gruppe und das ausgewählte Assignment
\end{itemize}

\subsubsection*{\textit{checkAccess}}\label{checkAccessTGUI}
\subparagraph{Beschreibung}
\begin{itemize}
	\item[] \noindent\fbox{checkAccess()}
	\item[] Überprüft, ob man Lese- oder Schreibrechte auf das aktuelle Objekt hat.
\end{itemize}